\documentclass{article}
\usepackage[utf8]{inputenc}
\usepackage{fancyhdr}
\usepackage[left=2cm, right=2cm, top=2cm, bottom=2cm]{geometry}
\usepackage{parskip}
\usepackage{algpseudocode}
\usepackage{color}
\usepackage{wrapfig}
\usepackage{graphicx}
\usepackage{amsmath}
\usepackage{amssymb}

\pagestyle{fancy}

\begin{document}

\lhead{LINMA2710 - Project 2}
\rhead{Adrien Banse (12501700)}
\cfoot{}

\section{Problem considered}
For  some function $u: \Omega \to \mathbb{R}$, consider the following differential equation:
\begin{equation}
\label{PDE}
	\frac{\partial u(x, y, t)}{\partial t} = \alpha \left( \frac{\partial^2 u(x, y, t)}{\partial x^2} + \frac{\partial^2 u(x, y, t)}{\partial y^2} \right).
\end{equation}
Equation~\eqref{PDE} is a special case of the diffusion problem considered in the statement, with $D(x, y) = \alpha$ and $f(x, y) = 0$ for all $x, y$. We consider the following values for the problem: 
\begin{itemize}
	\item $\Omega = [0, 300] \times [0, 50] \times [0, 50]$,
	\item $\alpha = 2$.
\end{itemize}
Note that a physical way to interpret Equation~\eqref{PDE} is to consider the diffusion of \emph{heat} on a 2D plate of dimensions 50[m] $\times$ 50[m] during 300[s]. $u(t, x, y)$ is thus the temperature in [K] at time $t$[s] at the point at $x$[m] of the origin of the plate in one direction and $y$[m] of the origin in the other direction\footnote{[m] stands for \emph{meters}, [s] for \emph{seconds} and [K] for \emph{Kelvin}.}. 

Consider that, with the same notations as in the statement, we choose somes values of $\Delta x$, $\Delta y$, then the explicit scheme is numerically stable if \cite[Equations (3.5), (3.6)]{morton_mayers_2005}:
\begin{equation}
	yo
\end{equation}

\section{Results}

\textcolor{red}{TODO}

\bibliographystyle{plain} % We choose the "plain" reference style
\bibliography{ref} % Entries are in the refs.bib file

\end{document}
